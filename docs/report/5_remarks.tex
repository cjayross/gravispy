\documentclass{standalone}

\begin{document}

As brought up in Subsection \ref{sec:method:implement}, there is an issue with our current methods that results in visible tearing at the $\pm\pi/2$ meridians of the image.
However, since this image is intended to be used to texture a sphere in the final OpenGL implementation and that the observer in the simulation---who will be sitting at the sphere's center---will not have a field of view larger than $\pi$, this problem can perhaps be hidden from the end user.
Nonetheless, this may be due to mistakes we have yet to see in our underlying geometry that may make our lensing less robust or may introduce bugs latter on.

Aside from this, further updates intend to also implement gravitational blueshift of light that occurs when the observer is significantly closer to the singularity than the sources.
Another, much more significant problem that our code overlooks, is that of the existence of hidden images.
Hidden images represent the degeneracies of the lensing map, meaning that a single point on the observer's sky maps to more than one location on the source sky.
The result of this is going to be an amplification of the luminosity at specific locations on the observer's sky and is the phenomena that most accentuates the presence of Einstein rings.


\end{document}
