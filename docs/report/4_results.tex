\documentclass{standalone}

\begin{document}
\begin{figure*}
  \caption{\label{fig:example1}
    Final result of applying the explicit Schwarzschild lensing map on a sample image using a Schwarzschild space-time defined by $M=1.0$ and $r_O=10M$.
    In this example, one can easily see the problematic image tearing at the $\pi/2$ and $-\pi/2$ meridians.
  }
  \includegraphics[width=\textwidth]{example_lens1}
\end{figure*}

\begin{figure*}
  \caption{\label{fig:example2}
    Second example of a generated Schwarzschild lensing map using the same space-time as in Fig.(\ref{fig:example1}) but at a radius of $r_O=\SI{2.5}{\kilo\meter}$, which is significantly closer to the event horizon.
    The accuracy of this result hasn't yet been scrutinized, however the overall converging of the observer's sky to the point at $\phi=0$ is to be expected.
  }
  \includegraphics[width=\textwidth]{example_lens2}
\end{figure*}

\begin{figure*}
  \caption{\label{fig:lenscomparison}
    Comparison between the three levels of generality in the Schwarzschild lensing equation.
    The first graph accentuates the differences between the methods near the singularity, while the latter graph shows how the errors of the first decrease as the radius of the observer increases.
  }
  \begin{tabular}{c}
    \includegraphics[width=.75\textwidth]{sc_lensing_near}\\
    \includegraphics[width=.75\textwidth]{sc_lensing_far}
  \end{tabular}
\end{figure*}

\end{document}
