\documentclass{standalone}

\begin{document}

Gravitational lensing is the process by which the curvature of spacetime induced by the presence of a very massive and dense object distorts the straight line paths traveled by light.
This results in a warped perspective of background light sources that lie behind said object with respect to the observer.

The goal of our project is to model this process visually.
By which we mean to incorporate the lensing equations that have been solved by cosmologists such as \cite[Perlick]{gen_lens} and \cite[Frittelli, et al.]{sc_lens} into OpenGL so that we can create interactive, graphical visualizations of the distortions in (ideally) real-time.

Although we managed to accomplish the numerical methods and calculations for the lensing equations, we have not yet been able to finish it's incorporation into OpenGL.
However, we have managed to create Gravispy's first simulation by applying our calculated distortions onto an image supplied by the user as discussed in Section \ref{sec:results}.

\end{document}
