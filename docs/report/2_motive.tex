\documentclass{standalone}

\begin{document}

As documented by \cite[Perlick]{gen_lens}, the process of analyzing the exact lensing of spherical spacetimes amounts to representing light that travels backwards in time from the point of their observation.
What this means is that observed light is modeled to originate from the observer themself and propagates out to the source from which it was created.
This is exactly the assumptions made in the raytracing algorithms applied in computer graphics.
As such, the prospect of modeling the phenomena of gravitational lensing is an enticing one, however, most existing implementations of these lensing models are much more tailored toward scientific representations rather than a more intuitive, layman representation of such a beautiful phenomena.

In terms of the course, the necessary problems that are presented with this task happen to also mirror the very subjects covered in class, including root finding and numerical integration.
And yet, this happens to be true in such a way that is not too overbearing for a semester project but certainly challenging enough to require us to expand our understanding of the subjects involved in order to accomplish our goal.

\end{document}
