% learn about REVTeX at https://journals.aps.org/revtex
% useful references:
%   https://cdn.journals.aps.org/files/revtex/summary4-1.pdf
%   ftp://ftp.dante.de/tex-archive/info/latex-refsheet/LaTeX_RefSheet.pdf
%   http://www-groups.mcs.st-andrews.ac.uk/~alanc/pub/c_tikzref/c_tikzref.pdf
\documentclass[
%draft,
aps,prd,
preprint,
onecolumn,
12pt,
amsmath, amssymb,
secnumarabic,
]{revtex4-1}

\usepackage[utf8]{inputenc}
\usepackage[english]{babel}
\usepackage{hyperref}
\usepackage[dvipsnames]{xcolor}
\usepackage{standalone}
\usepackage{multirow}
\usepackage{longtable}
\usepackage{graphicx}
\usepackage{tikz}
\usepackage{minted}
\usepackage{siunitx}
\usepackage[italicdiff]{physics}

\hypersetup{%
colorlinks=true,
linkcolor=RoyalBlue,
urlcolor=RoyalBlue,
citecolor=RoyalBlue,
}
\usetikzlibrary{decorations.markings}
\usetikzlibrary{calc}
\graphicspath{{../png/}}
\usemintedstyle{vs}
\definecolor{LightGray}{gray}{.9}
\setminted[python]{%
bgcolor=LightGray,
mathescape=true,
python3=true,
baselinestretch=1,
}

\newcommand{\comment}[1]{}
\newcommand{\python}[1]{\textcolor{BrickRed}{\texttt{#1}}}

\begin{document}

\title{Gravispy: gravitational simulations in open source software}
\homepage{https://github.com/cjayross/gravispy}
\author{Spence Norwood}
\author{Kellen O'Keefe}
\author{Calvin Ross}
\affiliation{University of Texas at Dallas, Physics Department}
\begin{abstract}
Gravispy is designed to visualize the effects of gravitational lensing.
The goals of the project is to calculate a massive object's effect on spacetime using the Schwarzschild metric and modeling the resulting warped perspective in three dimensions.
By using the Gauss-Kronrod quadrature formula for integration and the Brent-Dekker method for finding the roots of equations, the null-geodesics representing the trajectories of light are calculated numerically at radii near the event horizon of a black hole.
These values are then used to create a mapping between the light of the unperturbed state supplied by the user and the final lensed state.
With this map, a resultant image is produced and is projected onto a sphere using OpenGL to provide a panoramic view of the environment.
The results of this project have been generally successful.
Gravitational lensing has been correctly modeled---with the exception of a small shearing of the output image---and the lensing effect is viewable in a three dimensional model as intended.
Initial goals were to make the simulation dynamic, but currently only a static simulation is supported.
Future improvements to this project could be made by implementing this functionality, and by increasing the physical accuracy of the simulation by modeling the effect of blueshift and the existence of hidden images.
\end{abstract}
\maketitle

\section{Introduction\label{sec:intro}}
\documentclass{standalone}

\begin{document}

Hello

\end{document}


\section{Motivation\label{sec:motive}}
\documentclass{standalone}

\begin{document}

As documented by \cite[Perlick]{gen_lens}, the process of analyzing the exact lensing of spherical spacetimes amounts to representing light that travels backwards in time from the point of their observation.
What this means is that observed light is modeled to originate from the observer themself and propagates out to the source from which it was created.
This is exactly the assumptions made in the raytracing algorithms applied in computer graphics.
As such, the prospect of modeling the phenomena of gravitational lensing is an enticing one, however, most existing implementations of these lensing models are much more tailored toward scientific representations rather than a more intuitive, layman representation of such a beautiful phenomena.

In terms of the course, the necessary problems that are presented with this task happen to also mirror the very subjects covered in class, including root finding and numerical integration.
And yet, this happens to be true in such a way that is not too overbearing for a semester project but certainly challenging enough to require us to expand our understanding of the subjects involved in order to accomplish our goal.

\end{document}


\section{Methods\label{sec:method}}
\documentclass{standalone}
\usepackage{multirow}

\begin{document}

\begin{longtable}{cc}
  \begin{tabular}[t]{cc}
    \includegraphics[width=\textwidth]{earth}
    & {\huge $\Longrightarrow$}
  \end{tabular}
\end{longtable}

\end{document}


\section{Results\label{sec:results}}
\documentclass{standalone}

\begin{document}
\begin{figure*}
  \caption{\label{fig:earthsample}
    Sample image used to test the application of a generated lensing map.
  }
  \includegraphics[width=\textwidth]{earth}
\end{figure*}

\begin{figure*}
  \caption{\label{fig:example1}
    Final result of applying the explicit Schwarzschild lensing map on the sample image shown in Fig.(\ref{fig:earthsample}) using a Schwarzschild space-time defined by $M=\SI{1}{\kilo\meter}$ and $R_O=\SI{10}{\kilo\meter}$.
    In this example, one can easily see the problematic image tearing at the $\pi/2$ and $-\pi/2$ meridians.
  }
  \includegraphics[width=\textwidth]{example_lens1}
\end{figure*}

\begin{figure*}
  \caption{\label{fig:example2}
    Second example of a generated Schwarzschild lensing map using the same space-time as in Fig.(\ref{fig:example1}) but at a radius of $R_O=\SI{2.5}{\kilo\meter}$, which is significantly closer to the event horizon.
    The accuracy of this result hasn't yet been scrutinized, however the overall converging of the observer's sky to the point at $\phi=0$ is to be expected.
  }
  \includegraphics[width=\textwidth]{example_lens2}
\end{figure*}

\begin{figure*}
  \caption{\label{fig:lenscomparison}
    Comparison between the three levels of approximation in the Schwarzschild lensing equation.
    The first graph indicates the differences near the singularity, while the latter shows that the amount of error decreases significantly as the radius of the observer increases.
  }
  \begin{tabular}{c}
    \includegraphics[width=.75\textwidth]{sc_lensing_near}\\
    \includegraphics[width=.75\textwidth]{sc_lensing_far}
  \end{tabular}
\end{figure*}

\end{document}


\section{Discussion and Conclusion}
\documentclass{standalone}

\begin{document}

As brought up in Subsection \ref{sec:method:implement}, there is an issue with our current methods that results in visible tearing at the $\pm\pi/2$ meridians of the image.
However, since this image is intended to be used to texture a sphere in the final OpenGL implementation and that the observer in the simulation---who will be sitting at the sphere's center---will not have a field of view larger than $\pi$, this problem can perhaps be hidden from the end user.
Nonetheless, this bug may be due to mistakes we have yet to see in our underlying geometry that may make our lensing less robust or may introduce more bugs latter on.

Aside from this, further updates intend to also implement gravitational blueshift of light that occurs when the observer is significantly closer to the singularity than the sources.
Another, much more significant problem that our code overlooks is that of the existence of hidden images.
Hidden images represent the degeneracies of the lensing map, meaning that a single point on the observer's sky maps to more than one location on the source sky.
The result of this is going to be an amplification of the luminosity at specific locations on the observer's sky and is the phenomena that most accentuates the presence of Einstein rings.

\end{document}


\nocite{*}
\bibliography{report}

\end{document}
